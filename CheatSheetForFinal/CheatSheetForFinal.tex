% !TEX program = pdflatex
% 固体物理 Cheat Sheet
\documentclass[UTF8,10pt,a4paper]{article}
\usepackage[UTF8,scheme=plain,linespread=.1]{ctex}
\usepackage[margin=.1in]{geometry}
\usepackage{multicol}
\setlength{\columnseprule}{.2pt}
\usepackage{amsmath,amssymb,amsthm,bm}
\allowdisplaybreaks[4]
\providecommand{\abs}[1]{\lvert#1\rvert}
\providecommand{\re}{\mathrm{Re}}
\usepackage{ulem}
\usepackage[version=4]{mhchem}
\begin{document}
\tiny
\begin{multicols}{2}
\noindent\textbf{Part1不考虑微观结构的固体物理:早期固体物理Chap2固体热容:玻尔兹曼,爱因斯坦\&德拜}\\
\textbf{杜隆-珀蒂(Dulong-Petit)定律}:许多固体热容$C=3k_B\text{每原子或}3R$,常温下符合较好,$C\downarrow$随$T\downarrow$\\
\textbf{玻尔兹曼(Boltzmann)的模型}:视各原子处于与邻近原子相互作用而形成的简谐势阱中,由能量均分定理,$T$下各(能量为平方形式的)自由度平均能量为$k_BT/2$,单原子平均能量$\langle E\rangle=3k_BT/2+3k_BT/2$,热容$C=\mathrm{d}\langle E\rangle/\mathrm{d}T=3k_B$,无法解释某些固体热容的偏离和随$T$的变化\\
\textbf{爱因斯坦(Einstein)模型}:视原子为本征频率相等的谐振子,1维情况:单原子能级$E_n=\hbar\omega(n+1/2)$,配分函数$Z_1=\sum_n\exp(-\beta E_n)=\frac{\exp(-\beta\hbar\omega/2)}{1-\exp(-\beta\hbar\omega)}$,单原子平均能量$\langle E_1\rangle=\frac{\sum_n\hbar\omega(n+1/2)\exp[-\beta\hbar\omega(n+1/2)]}{\sum_n\exp[-\beta\hbar\omega(n+1/2)]}=-\frac{1}{Z}\frac{\partial Z}{\partial\beta}=\hbar\omega[n_B(\beta\hbar\omega)+\frac{1}{2}]$,其中玻色(Bose)填充因子$n_B(x)=\frac{1}{\exp(x)-1}$(相当于视晶格振动为一种准粒子-声子,因其可湮灭和产生,故为玻色子),热容$C_1=\frac{\partial\langle E_1\rangle}{\partial T}=k_B(\beta\hbar\omega)^2\frac{\exp(\beta\hbar\omega)}{[\exp(\beta\hbar\omega)-1]^2}$;3维情况:能级$E_{n_x,n_y,n_z}=\hbar\omega[(n_x+1/2)+(n_y+1/2)+(n_z+1/2)]$,配分函数$Z=Z_1^3$,平均能量$\langle E\rangle=3\langle E_1\rangle$,热容$C=3C_1$;高温下,$C\approx 3k_B$,低温下,$C\approx 0$;金刚石劲度系数$\kappa$大,$\omega=\sqrt{\kappa/m}$大,能级差大,激发难,故室温下$C$小\\
\textbf{德拜(Debye)模型}:认为各振子本征频率不同;假设线性色散关系:$\omega=vk$,平均能量$\langle E\rangle=3\sum_{\omega}\hbar\omega[n_B(\beta\hbar\omega)+1/2]=\int_0^{\omega_D}d\omega\,g(\omega)\hbar\omega[n_B(\beta\hbar\omega)+\frac{1}{2}]$,其中$[\omega,\omega+d\omega]$范围内状态数$=g(\omega)\,d\omega=3\frac{dk_x\,dk_y\,dk_z}{(2\pi/L)^3}=3\frac{L^3}{(2\pi)^3}4\pi k^2\,dk=\frac{3L^3}{2\pi^2}\frac{\omega^2}{v^3}d\omega\Rightarrow$态密度$g(\omega)=\frac{3L^3}{2\pi^2}\frac{\omega^2}{v^3}=9N\frac{\omega^2}{\omega_D^3}$,其中德拜频率$\omega_D$满足$\int_0^{\omega_D}d\omega\,g(\omega)=3N$($3$维情况有$3N$个振动模式)$\Rightarrow\omega_D^3=6\pi^2nv^3$;低温下,$n_B(\omega_D)\approx 0$,故积分上限可换为$\infty$,令$x=\beta\hbar\omega$,$\langle E\rangle=\frac{9N\hbar}{\omega_D^3(\beta\hbar)^4}\int_0^{\infty}dx\frac{x^3}{e^x-1}+E_0=9N\frac{(k_BT)^4}{(\hbar\omega_D)^3}\frac{\pi^4}{15}+E_0$,热容:$C=\frac{\partial\langle E\rangle}{\partial T}=Nk_B\frac{(k_BT)^3}{(\hbar\omega)^3}\frac{12\pi^4}{5}$(\textbf{德拜$T^3$率});高温下,$\frac{1}{e^x-1}\approx\frac{1}{x}$,$\langle E\rangle=k_BT\int_0^{\omega_D}d\omega\,g(\omega)+E_0=3k_BTN+E_0$,$C=3k_BN$;因线性色散关系假设过简单,中间温区有偏差;对低温下导体,$C=\gamma T^3+\beta T$,首项由电子贡献;1维情况:$g(\omega)\,\mathrm{d}\omega=\frac{\mathrm{d}k}{2\pi/L}=\frac{L}{2\pi}\frac{\mathrm{d}\omega}{v}$;2维情况:$g(\omega)\,\mathrm{d}\omega=\frac{\mathrm{d}k_x\,\mathrm{d}k_y}{(2\pi/L)^2}=\frac{L^2}{(2\pi)^2}2\pi k\,\mathrm{d}k=\frac{L^2}{2\pi}\frac{\omega}{v^2}\,\mathrm{d}\omega$\\%存疑:3维情况的态密度是否有系数3,若是,则2维情况的态密度应有系数2
\rule{\columnwidth}{.2pt}\\
\textbf{Chap3金属中的电子:德鲁德(Drude)理论}:三假设:平均碰撞时间$\tau$,$\mathrm{d}t$内碰撞概率$\mathrm{d}t/\tau$,证:在$t\sim t+\mathrm{d}t$内碰撞概率$P(t)=\lim\limits_{n\rightarrow\infty}(1-\frac{t/n}{\tau})^n\frac{\mathrm{d}t}{\tau}=e^{-t/\tau}\frac{\mathrm{d}t}{\tau}$,故$\int_0^{\infty}tP(t)\,\mathrm{d}t=\tau$;碰后平均速度$=0$;相邻两次碰撞间仅受外力驱动,$\frac{\mathrm{d}\bm{p}}{\mathrm{d}t}=\bm{F}=q\bm{E}+q\bm{v}\times\bm{B}$;$t$时刻动量$p(t)$,$t+\mathrm{d}t$时刻动量$\bm{p}(t+\mathrm{d}t)=[\bm{p}(t)+\bm{F}\,\mathrm{d}t](1-\frac{\mathrm{d}t}{\tau})+0\cdot\frac{\mathrm{d}t}{\tau}\Rightarrow\frac{\mathrm{d}p}{\mathrm{d}t}=-\frac{\bm{p}}{\tau}+\bm{F}$;$-\frac{\bm{p}}{\tau}$-阻尼项,若$\bm{F}=0$,则$\frac{\mathrm{d}\bm{p}}{\mathrm{d}t}=-\frac{\bm{p}}{\tau}\Rightarrow\bm{p}(t)=\bm{p}(0)e^{-t/\tau}$,动量指数衰减\\
\textbf{电导率}:当在外加电场$\bm{E}$下达稳定电流,$0=\frac{\mathrm{d}\bm{p}}{\mathrm{d}t}=-\frac{\bm{p}}{\tau}-eE\Rightarrow\langle\bm{p}\rangle=-e\bm{E}\tau$,电流密度$\bm{j}=-ne\bm{v}=\frac{ne^2\tau}{m}\bm{E}=\sigma\bm{E}$,其中\textbf{电导率}$\sigma=\frac{ne^2\tau}{m}$;对金属,$T\uparrow\Rightarrow\tau\downarrow\Rightarrow\sigma\downarrow$;对半导体,$T\uparrow\Rightarrow n\uparrow\Rightarrow\sigma\uparrow$\\
\textbf{霍尔(Hall)效应}:再加磁场$\bm{B}=B\bm{z}$,有$0=\frac{\mathrm{d}\bm{p}}{\mathrm{d}t}=-\frac{\bm{p}}{\tau}-e(\bm{E}+\bm{v}\times\bm{B})\Rightarrow\bm{E}=-\frac{m\bm{v}}{\tau e}-\bm{v}\times\bm{B}=-\frac{m\bm{v}}{\tau e}-(v_y\hat{x}-v_z\hat{y})B$,又$\bm{v}=-\frac{\bm{j}}{ne}$,故$\left\{\begin{smallmatrix}
    E_x=\frac{mj_x}{ne^2\tau}+B\frac{j_y}{ne}\\
    E_y=\frac{mj_y}{ne^2\tau}-B\frac{j_x}{ne}\\
    E_z=\frac{mj_z}{ne^2\tau}
\end{smallmatrix}\right.$,即$\bm{E}=\left[\begin{smallmatrix}
    \frac{m}{ne^2\tau}&\frac{B}{ne}&0\\
    -\frac{B}{ne}&\frac{m}{ne^2\tau}&0\\
    0&0&\frac{m}{ne^2\tau}
\end{smallmatrix}\right]\bm{j}$,霍尔电导率$\rho_{xy}=\frac{B}{ne}$,霍尔系数$R_H=\frac{\rho_{xy}}{B}=\frac{1}{ne}$;可用霍尔元件测$\bm{B}$或$n$\\
\textbf{热输运}:高温$T_1$区向低温$T_2$区的热流$j_h=\frac{1}{2}n(\langle v_{1x}\rangle cT_1-\langle v_{2x}\rangle cT_2)\approx\frac{1}{2}n\langle v_x\rangle c\,\mathrm{d}T=$(冷热交接区宽度$=2$倍平均自由程)$\frac{1}{2}n\langle v_x\rangle c\frac{\mathrm{d}T}{\mathrm{d}x}(2\langle v_x\rangle\tau)=n\langle v_x\rangle^2n\tau c\frac{\mathrm{d}T}{\mathrm{d}x}=\frac{1}{3}\langle v\rangle^2n\tau c\frac{\mathrm{d}T}{\mathrm{d}x}\approx\frac{1}{3}(\frac{1}{2}m\langle v^2\rangle)\frac{2}{m}n\tau c\frac{\mathrm{d}T}{\mathrm{d}x}\approx\frac{1}{3}(\frac{3}{2}k_BT)\frac{2}{m}n\tau(3k_B/2)\frac{\mathrm{d}T}{\mathrm{d}x}=\frac{3}{2}\frac{k_B^2T}{m}n\tau\frac{\mathrm{d}T}{\mathrm{d}x}=\kappa\frac{\mathrm{d}T}{\mathrm{d}x}$,其中\textbf{热导率}$\kappa=\frac{3}{2}\frac{k_B^2T}{m}n\tau$,$c$-单粒子热容;\textbf{洛伦兹(Lorenz)数}:$L=\frac{\kappa}{T\sigma}=\frac{3}{2}(\frac{k_B}{e})^2$,与金属种类无关(维德曼–夫兰兹(Wiedmann-Franz)定理);以上推导不严谨,严重高估热容($c\ll\frac{3}{2}k_B$),低估速度(忽略电子热运动,$\langle v\rangle^2\ll\langle v^2\rangle$),歪打正着,严谨推导应用费米-狄拉克(Fermi-Dirac)统计;用类似方法推导塞贝克(Seeback)系数可见上述错误,\textbf{塞贝克效应}:温差导致电势差,\textbf{塞贝克系数}$S=\frac{\mathrm{d}V}{\mathrm{d}T}=\frac{E\,\mathrm{d}x}{\mathrm{d}T}=\frac{E}{dT/dx}$,热漂移$J_h=\frac{1}{2}n(v_1-v_2)=\frac{1}{2}n\frac{\mathrm{d}v}{\mathrm{d}x}(2v\tau)=n\tau v\frac{\mathrm{d}v}{\mathrm{d}x}=n\tau\frac{\mathrm{d}(mv^2/2)}{m\,\mathrm{d}x}=n\tau\frac{\mathrm{d}(k_BT/2)}{m\,\mathrm{d}x}=\frac{n\tau k_B}{2m}\frac{\mathrm{d}T}{\mathrm{d}x}$,\textbf{电漂移}$J_e=nv=n\frac{eE}{m}\tau$,两者平衡,$J_h=J_e\Rightarrow S=\frac{k_B}{2e}\sim 10^{-4}\text{V}/\text{K}$,与实验值$\sim 10^{-6}\text{V}/\text{K}$不符\\
\rule{\columnwidth}{.2pt}\\
\textbf{Chap4金属中的更多电子:索末菲(Sommerfeld)自由电子理论}:由\textbf{费米-狄拉克分布},$n_F(\beta(\epsilon-\mu))=\frac{1}{\exp[\beta(\epsilon-\mu)]+1}$,其中$\mu$-化学势:体系增/减$1$个粒子引起的能量变化,满足$N=2\sum\limits_k\frac{1}{\exp[\beta(\epsilon(k)-\mu)]+1}$,$\lim\limits_{T\rightarrow 0}n_F=$阶跃函数,$\lim\limits_{T\rightarrow 0}\frac{\mathrm{d}n_F}{\mathrm{d}T}=\delta(\epsilon-\mu)$;低温下,$N=2\frac{V}{(2\pi)^3}\iiint\mathrm{d}k_x\,\mathrm{d}k_y\,\mathrm{d}k_z=\frac{2V}{(2\pi)^3}\frac{4}{3}\pi k_{\max}^3\Rightarrow$\textbf{费米波矢}$k_F=k_{\max}=(\frac{3\pi^2N}{V})^{1/3}$,\textbf{费米速度}$v_F=\frac{\hbar k_F}{m}\sim 1\%c$,\textbf{费米能量}$\epsilon_F=\mu(T=0)=\hbar^2k_F^2/2m=\hbar^2(3\pi^2n)^{2/3}/2m$($n\sim 10^{22+6}\text{m}^{-3}$,$m\sim 10^{-30}$kg)$\sim 7e\text{V}\sim 80000\text{K}$(\textbf{费米温度})$\gg$室温$300\text{K}\sim 0.025e\text{V}$,热激发仅在费米海的表面掀起微微的涟漪-升温很难明显改变化学势\\
\textbf{电子贡献热容}:自由电子总能量$\langle E\rangle=V\int\mathrm{d}\epsilon\,\epsilon n_F(\epsilon)g(\epsilon)$,其中$k=\sqrt{2m\epsilon}/\hbar\Rightarrow$单位体积$[\epsilon,\epsilon+\mathrm{d}\epsilon]$范围内状态数$=g(\epsilon)\,\mathrm{d}\epsilon=\frac{2}{(2\pi)^3}4\pi k^2\,\mathrm{d}k=\frac{(2m)^{3/2}}{2\pi^2\hbar^3}\epsilon^{1/2}\,\mathrm{d}\epsilon\Rightarrow$态密度$g(\epsilon)=\frac{3n}{2\epsilon_F}(\frac{\epsilon}{\epsilon_F})^{1/2}$;非高温下,$\mu\approx\epsilon_F$,费米面附近可激发电子数$\sim Vg(\epsilon_F)\cdot k_BT$,单电子热激发能$\sim k_BT$,总能量:$E\sim Vg(\epsilon_F)(k_BT)^2$,电子贡献热容$C=\frac{\mathrm{d}E}{\mathrm{d}T}\sim Vg(\epsilon)k_B^2T$\\
\textbf{泡利(Pauli)顺磁}:电子磁矩$\bm{m}=g\mu_B\bm{\sigma}$,其中玻尔(Bohr)磁子$\mu_B=\frac{e\hbar}{2m_e}$,对自由电子,自旋$\sigma=\frac{1}{2}$,朗德$g$因子$g=-2.002$,磁场$\bm{B}$下能量$\epsilon=-\bm{m}\cdot\bm{B}$,单位体积内$\bm{\sigma}$与$\bm{B}$同向的电子态密度相对费米面上移$\sim\mu_BB$,这部分电子数量减少$\sim\mu_BBg(\epsilon_F)/2$,反向的态密度下移$\sim\mu_BB$,数量增加$\sim\mu_BBg(\epsilon_F)/2$,相当于这么多$\bm{\sigma}$与$\bm{B}$同向的电子换成反向,每个造成磁矩变化$2\mu_B$,系统磁化强度$M=\mu_BB\frac{g(\epsilon_F)}{2}\cdot 2\mu_B$,磁化率$\chi_p=\frac{\partial M}{\partial H}=\mu_0\frac{\partial M}{\partial B}=\mu_0\mu_B^2g(\epsilon_F)$;此处忽略费米-狄拉克分布,故结论与温度无关\\
\textbf{德鲁德输运模型的缺陷}:漂移速度$\sim 10^{-5}\text{m}/\text{s}\ll v_F$,平均自由程$v_F\tau=10^{-6}\text{m}\gg$晶格常数;因忽视晶体结构无法解释霍尔效应积累电荷方向;无法解释铁磁性\\
\rule{\columnwidth}{.2pt}\\
\textbf{Part2材料结构Chap5元素周期表}\\
\textbf{主量子数}:$n=1,2,\cdots$,对应轨道最多电子数$2n^2$,\textbf{角量子数}:$l=0,\cdots,n-1$,对应序号s,p,d,f,最多电子数$2[(l+1)^2-l^2]$,\textbf{磁量子数}:$l_z=-l,\cdots,l$,\textbf{自旋量子数}:$\pm 1/2$\\
\textbf{构造原理(Aufbau principle)}:优先填充低能轨道;\textbf{马德隆规则(Madelung's rule)}:按$n+l$增序填充,相同$n+l$按$n$增序填充,即1s,2s,2p,3s,3p,4s,3d,4p,5s,4d,5p,6s,4f,$\cdots$;少数元素如\ce{Cu}([\ce{Ar}]4s$^1$3d$^{10}$)及成键时或违反马德隆规则\\
左$\rightarrow$右,下$\rightarrow$上,电离能(失一电子所需能量)$\uparrow$,电子亲和能(得一电子放出能量)$\uparrow$\\
\rule{\columnwidth}{.2pt}\\
\textbf{Chap6化学键}:\textbf{离子键},\textbf{共价键},\textbf{金属键},\textbf{范德华力},\textbf{氢键}\\
\textbf{离子键}:电子转移形成阴阳离子,相互吸引,一般由电负性差距较大的元素组成;键能$\Delta E=\Delta E_1+\Delta E_2$,其中$\Delta E_1=$阳离子的电离能$+$阴离子的电子亲和能,$\Delta E_2$-阴阳离子的内聚能;电负性:对电子的吸引能力,密立根(Mulliken)电负性$=($电子亲和能+电离能$)/2$,鲍林(Pauling)电负性(通用)$=0.187\times 2\times$密立根电负性$+0.17$;离子晶体特点:硬,脆,不导电,透明,溶于水\\
\textbf{共价键}:共享电子,类比一维无限深势阱,基态$k=\frac{\pi}{L}$,能量$E=\frac{\hbar^2\hbar^2}{2m}=\frac{\hbar^2\pi^2}{2mL^2}$,共享电子而成键后$L\uparrow$,波函数扩展,$E\downarrow$;可用紧束缚模型,以H$_2^+$为例,电子哈密顿$H=\frac{p^2}{2m}-\frac{e^2}{4\pi\epsilon_0 r_1}-\frac{e^2}{4\pi\epsilon_0r_2}=K+V_1+V_2$,近似认为状态为H原子基态的叠加,$\lvert\psi\rangle=\phi_1\lvert 1\rangle+\phi_2\lvert 2\rangle$,其中$\lvert n\rangle$-第$n$个H原子的1s轨道,平均能量$\langle E\rangle=\frac{\langle\psi\rvert H\lvert\psi\rangle}{\langle\psi\vert\psi\rangle}=\frac{\sum_{ij}\phi_i\phi_j^*H_{ij}}{\sum_i\phi_i\phi_i^*}$,其中$H_{ij}=\langle i\rvert H\lvert j\rangle$,由变分原理,$\frac{\delta\langle E\rangle}{\delta\phi_i^*}=\frac{\sum_i\phi_iH_{ij}}{\sum_i\phi_i\phi_i^*}-\frac{\sum_{ij}\phi_i\phi_j^*H_{ij}}{(\sum_i\phi_i\phi_i^*)^2}\phi_i=0\Rightarrow\sum_i\phi_iH_{ij}=\langle E\rangle\phi_j\Rightarrow H(\phi_1\,\phi_2)^T=\langle E\rangle(\phi_1\,\phi_2)^T$,其中$H_{11}=\langle 1\rvert(K+V_1+V_2)\lvert 1\rangle=\epsilon_{\text{1s}}+\langle 1\rvert V_2\lvert 1\rangle=\epsilon_{\text{1s}}+V_{\text{cross}}=H_{22}$,$H_{12}=\langle 1\rvert(K+V_1+V_2)\lvert 2\rangle=\langle 1\rvert V_1\lvert 2\rangle=-t=H_{21}^*$,本征能量$\langle E\rangle=\epsilon_{\text{1s}}+V_{\text{cross}}\pm t$,本征态$\lvert\psi\rangle=\frac{1}{\sqrt{2}}(\lvert 1\rangle\pm\lvert 2\rangle)$;\textbf{共价化合物特点}:不溶于水,或为半导体\\
\textbf{范德华(Van der Waals)力}:本质:电偶极相互作用;兰纳-琼斯(Lennard-Jones)势:$V=Ar^{-12}-Br^{-6}$,吸引项(末项)-一原子偶极矩$\bm{p}_1$垂直两原子连线,在另一原子处有沿连线的电场$E=\frac{p_1}{4\pi\epsilon_0r^3}$,激发其偶极矩$p_1=\chi E$,两者相互作用势能$U=\frac{-p_1p_2}{4\pi\epsilon_0r^3}=\frac{-\chi p_1^2}{(4\pi\epsilon_0r^3)^2}$\\
\textbf{金属键}:类似胶水,使金属有延展性\\
\rule{\columnwidth}{.2pt}\\
\textbf{Chap7物质类型}:\textbf{晶体}:一般由原子构成,但也有分子晶体,如糖;\textbf{液晶};\textbf{非晶}(\textbf{多晶});\textbf{准晶}:不具平移对称性\\
\rule{\columnwidth}{.2pt}\\
\textbf{Part3$1$维固体的玩具模型Chap8压缩系数,声音\&热胀系数的$1$维模型}\\
视晶体中原子为谐振子,势能$u(x)=u(a)+u'(a)(x-a)+\frac{1}{2}u''(a)(x-a)^2+\cdots$,在平衡位置不受力,故$u'(a)=0$,$u(x)\approx u(a)+\frac{1}{2}u''(a)(x-a)^2$,受力$F=-\frac{\mathrm{d}u}{\mathrm{d}x}=-\kappa(x-a)$,其中劲度系数$\kappa=u''(x)$\\
\textbf{压缩系数}:增大单位压强导致体积收缩比例,$\beta=-\frac{1}{V}\frac{\partial V}{\partial P}$;\textbf{体弹模量}:导致体积收缩单位比例所需压强,$\beta^{-1}=-V\frac{\partial P}{\partial V}$(对固体)$>10^9$Pa;\textbf{杨氏模量}:导致长度伸长单位比例所需单位横截面上的拉力,$E=-\frac{F/A}{\Delta L/L_0}=-L_0\frac{P}{\Delta L}$,若拉伸时横截面积不变,$E=-V_0\frac{P}{\Delta V}=$体弹模量,对柱形气缸中的理想气体,$(P_0+\Delta P)(L_0-\Delta L)=PL_0\Rightarrow\Delta PL_0-P_0\Delta L\approx 0\Rightarrow E\approx P_0\sim 10^5$Pa;$1$维情况下,$\beta^{-1}=-L\frac{\partial F}{\partial L}=-Na\frac{\partial F}{\partial(Na)}=-a\frac{\partial F}{\partial a}=-a\kappa$,其中$a$-原子间距\\
\textbf{声速}:$v=\sqrt{\frac{1}{\rho\beta}}=$(对气体)$\sqrt{\frac{P}{\rho}}$(对空气)$\approx\sqrt{\frac{10^5\text{Pa}}{1\text{kg}/\text{m}^3}}\approx 300\text{m}/\text{s}$;证:$1$维连续介质其中一小段的动力学方程$\rho\mathrm{d}x\frac{\mathrm{d}^2A}{\mathrm{d}t^2}=F(x+\mathrm{d}x)-F(x)$,其中$F(x)=\kappa[A(x+\delta x)-A(x)]=\frac{A(x+\delta x)-A(x)}{\beta\,\delta x}=\frac{1}{\beta}\left.\frac{\partial A}{\partial x}\right\rvert_x\Rightarrow$波动方程$\frac{\partial^2A}{\partial t^2}=v^2\frac{\partial^2A}{\partial x^2}$,其中$v=\sqrt{\frac{1}{\rho\beta}}$\\
\rule{\columnwidth}{.2pt}\\
\textbf{Chap9$1$维单原子链的振动}:劲度系数为$\kappa$的键连接相邻的质量为$m$的原子,第$n$个原子的动力学方程$m\ddot{\delta x_n}=\kappa(\delta x_{n+1}-\delta x_n)+\kappa(\delta x_{n-1}-\delta x_n)=\kappa(\delta x_{n+1}+\delta x_{n-1}-2\delta x_n)$,设$\delta x_n=A\exp[i(\omega t-kna)]$,回代得$-m\omega^2=\kappa[\exp(-ika)+\exp(ika)-2]=2\kappa(\cos ak-1)=-4\kappa\sin^2(ka/2)\Rightarrow$\textbf{色散关系}:$\omega=2\sqrt{\frac{\kappa}{m}}\abs{\sin\frac{ak}{2}}$;短波有欠采样,故色散关系呈周期性,\textbf{第一布里渊区(first Brillouin zone)}:$[-\frac{\pi}{a},\frac{\pi}{a}]$,其中状态数$=\frac{2\pi}{a}/\frac{2\pi}{Na}=N$;长波极限下,$\omega\approx a\sqrt{\frac{\kappa}{m}}k$,波速$v=\frac{\omega}{k}=a\sqrt{\frac{\kappa}{m}}$,短波极限下,$k\rightarrow\frac{\pi}{a}$,群速度$v_g=\frac{\mathrm{d}\omega}{\mathrm{d}k}=0$;声子动量$p=\hbar(k\bmod\frac{2\pi}{a})$,声子碰撞时动量或转化为晶格动量\\
\rule{\columnwidth}{.2pt}\\
\textbf{Chap10$1$维双原子链的振动}:弹性系数为$\kappa_1,\kappa_2$的键交替连接质量为$m$的原子;基元中两种原子的动力学方程为$m\ddot{\delta x_n}=\kappa_2(\delta y_{n-1}-\delta x_n)+\kappa_1(\delta y_n-\delta x_n)$,$m\ddot{\delta y_n}=\kappa_1(\delta x_n-\delta y_n)+\kappa_2(\delta x_{n+1}-\delta y_n)$; 设$\delta x_n=A_x\exp[i(\omega t-kna)]$,$\delta y_n=A_y\exp[i(\omega t-nka)]$,其中$A_x,A_y$为复数,含振幅和相位,回代得$-m\omega^2\left[\begin{smallmatrix}
    A_x\\
    A_y
\end{smallmatrix}\right]=\left[\begin{smallmatrix}
    -\kappa_1-\kappa_2&\kappa_1+\kappa_2\exp(ika)\\
    \kappa_1+\kappa_2\exp(-ika)&-\kappa_1-\kappa_2
\end{smallmatrix}\right]\left[\begin{smallmatrix}
    A_x\\
    A_y
\end{smallmatrix}\right]$,对角化得\textbf{色散关系}:$\omega_{\pm}(k)=\sqrt{(\kappa_1+\kappa_2)/m\pm 1/m\sqrt{(\kappa_1+\kappa_2)^2-4\kappa_1\kappa_2\sin^2(ka/2)}}$,其中$\omega_-$-\textbf{声学模},$\omega_+$-\textbf{光学模}\\
长波极限下$\omega_{\pm}=\sqrt{\frac{\kappa_1+\kappa_2}{m}\left[1\pm\sqrt{1-\frac{4\kappa_1\kappa_2\sin^2(ka/2)}{(\kappa_1+\kappa_2)^2}}\right]}\approx\sqrt{\frac{\kappa_1+\kappa_2}{m}\left[1\pm\left(1-\frac{2\kappa_1\kappa_2\sin^2(ka/2)}{(\kappa_1+\kappa_2)^2}\right)\right]}$, $\omega_-\approx\sqrt{\frac{2\kappa_1\kappa_2}{m(\kappa_1+\kappa_2)}}\abs{\sin\frac{ka}{2}}\approx\sqrt{\frac{2\kappa_1\kappa_2}{m(\kappa_1+\kappa_2)}}\abs{\frac{ka}{2}}$($\cong$线性),$\omega_+=\sqrt{\frac{2(\kappa_1+\kappa_2)}{m}-\frac{2\kappa_1\kappa_2\sin^2(ka/2)}{m(\kappa_1+\kappa_2)}}$\\
\textbf{声速}:因声波波长长,$v_s=\lim\limits_{k\rightarrow 0}\frac{\mathrm{d}\omega_-}{\mathrm{d}k}=\sqrt{\frac{a^2\kappa_1\kappa_2}{2m(\kappa_1+\kappa_2)}}$或$v_s=\sqrt{\frac{1}{\beta\rho}}$,其中$\beta=\frac{1}{\kappa a}$,$\rho=\frac{2m}{a}$,$\kappa=\frac{\kappa_1\kappa_2}{\kappa_1+\kappa_2}$\\
对光学模,$\omega_{+,\max}=\omega_+(0)=\sqrt{2(\kappa_1+\kappa_2)/m}$,群速度$\frac{\mathrm{d}\omega_+}{\mathrm{d}k}(0)=0$\\% 光学模最大能量$\sim 0.2eV$
\textbf{对光学模的理解}:当光与物质相互作用但传播方向不变(拉曼散射),光子及其激发声子动/能量守恒,$\hbar(\omega_1-\omega_2)=\hbar c(k_1-k_2)=\hbar\omega_p$, $\hbar(k_1-k_2)=\hbar k_p\Rightarrow\omega_p=ck_p$,因$c$很大,仅光学模中有这种长波高能的激发,即仅光学模可能与光子相互作用,故得名(非绝对,若相互作用时光子传播方向改变,声学模也可能与光子相互作用(布里渊散射))\\
长波极限下,$\omega^2\left[\begin{smallmatrix}
    A_x\\
    A_y
\end{smallmatrix}\right]=\frac{\kappa_1+\kappa_2}{m}\left[\begin{smallmatrix}
    1&-1\\
    -1&1
\end{smallmatrix}\right]\left[\begin{smallmatrix}
    A_x\\
    A_y
\end{smallmatrix}\right]$,对声学模$A_x=A_y$,对光学模$A_x=-A_y$,相邻原子运动方向相反(故能量也更高),如相邻正负离子在电场中的运动,故更易与光子相互作用,故得名(此为对其第二种理解)\\
$N$个基元,故$N$个$k$,每个基元$2$个原子,故每个$k$对应$2$个模式;若每个基元中$M$个原子,则每个$k$对应$1$个声学模和$M-1$个光学模;对$3$维原子链,每个$k$对应$3$个声学模和$3(M-1)$个光学模\\
\textbf{简约布里渊区图示(reduced zone scheme)}中,声/光学模同在一布区内;\textbf{拓展布里渊区图示(extended zone scheme)}中,声学模在一布区内,光学模在二布区,一布区边界处$\omega_{\pm}=\sqrt{2\min/\max\{\kappa_1,\kappa_2\}/m}$,$\frac{\mathrm{d}\omega_{\pm}}{\mathrm{d}k}(\pm\pi/a)=0$,当$\kappa_1=\kappa_2$,两模色散关系曲线相连成单原子链色散关系,故可视双原子链色散关系为单原子链在扰动下的情况\\
爱因斯坦模型的色散关系更接近光学模,德拜模型的色散关系更接近声学模\\
\rule{\columnwidth}{.2pt}\\
\textbf{Chap11紧束缚链}:1D原子链,设第$n$个原子基态为$\lvert n\rangle$,因原子间距$\gg$波尔半径,$\langle n\vert m\rangle\approx\delta_{nm}$,系统状态$\lvert \Psi\rangle=\sum_n\phi_n\lvert n\rangle$,能量$E=\frac{\langle\Psi\rvert\hat{H}\lvert\Psi\rangle}{\langle\Psi\vert\Psi\rangle}=\frac{\sum_{mn}\phi_m^*H_{mn}\phi_n}{\sum_m\phi_m^*\phi_m}$,由变分原理,$\frac{\delta E}{\delta\phi_m^*}=\frac{\sum_nH_{mn}\phi_n}{\sum_m\phi_m\phi_m^*}-\frac{\sum_n\phi_m^*H_{mn}\phi_n}{(\sum_m\phi_m\phi_m^*)^2}\phi_m=0$,得薛方$\sum\limits_nH_{mn}\phi_n=E\phi_m$,$H_{mn}=E_0\delta_{mn}+\sum_{j\neq n}\langle m\rvert V_j\lvert n\rangle$,其中$E_0$-单原子基态能,$V_j$-电子与第$j$个原子间的势能,近邻近似下,$\sum_{j\neq n}\langle m\rvert V_j\lvert n\rangle\approx V_0(\text{if }m=n),-t(\text{if }m=n\pm 1),0(\text{otherwise})$,其中hopping能$t>0\uparrow$随原子间距$\downarrow$,故$H_{mn}=\varepsilon_0\delta_{mn}-t(\delta_{m,n-1}+\delta_{m,n+1})$,其中$\varepsilon_0=E_0+V_0$,周期性边界条件下$\hat{H}$为准三对角矩阵,设$\phi_n=\exp(-ikna)/\sqrt{N}$,代入薛方得$E=\varepsilon_0-2t\cos ka$,类似1D原子链的色散关系\\
\textbf{能带}:电子本征能量范围;该原子链带宽$4t$,能带中心$\varepsilon_0$,本征能量一半$\uparrow$一半$\downarrow$,类似\ce{H_2^+};若能带未全占据,则总能量$\downarrow$随原子间距$\downarrow$(金属键);若每个原子$1$个价电子,因电子是费米子且自旋可取$\pm$,故仅填充下半能带,在电场下填充区偏移得电流,为导体;若$2$个价电子,则填满能带,无法载流,为能带绝缘体(非绝对,若电子间排斥能$>t$,则莫特(Mott)绝缘);若另考虑激发态或多原子链,则有多能带;如双原子链仅考虑基态,则有$2$条能带,其间或有带隙,若各原子均$1$个价电子,则仅填满较低一能带,仅强场下有电流,半导体\\
长波极限下(带底),$E\approx ta^2k^2+$const,平方项类比自由电子动能,$ta^2k^2=\hbar^2k^2/2m^*$,有效质量$m^*=\hbar^2/2ta^2$\\
\rule{\columnwidth}{.2pt}\\
\textbf{Part4晶体的结构Chap12晶体结构}\\
\textbf{晶格(lattice)}:由一组独立晶格基矢(primitive lattice vectors,不唯一)的所有整数倍线性组合构成的点集,$\{\bm{R}=l\bm{a}_1+m\bm{a}_2+n\bm{a}_3=[lmn]\vert l,m,n\in\mathbb{Z}\}$;蜂窝状的石墨烯非晶格,因其部分线性组合对应的点上无原子,但按现代定义是晶格但非Bravais格子;\textbf{Bravais格子}:各原子所处环境均相同\\
\textbf{晶胞(unit cell)}:晶格的重复单元(有面/体积);\textbf{\uline{原}初晶\uline{胞}(primitive unit cell)}:最小重复单元(对应一格点);\textbf{Wigner-Seitz原胞}:一格点与其最近的各格点连线的中垂线/面所围原胞;\textbf{基元(basis)}:每个格点的内容\\
空间利用率(packing fraction)$=$(假设原子为相互接触的小球)原子占用体积$/$晶体体积;\textbf{简单立方(primitive cubic lattice)}:$\pi/6$(仅一种单质-钋\ce{Po});\textbf{体心立方(\uline{b}ody \uline{c}entered \uline{c}ubic)}:$\sqrt{3}\pi/8$(\ce{Fe});\textbf{面心立方(\uline{f}ace-\uline{c}entered) cubic}:$\sqrt{2}\pi/6$,等价ABC型的六角密排(hexagonal close packing)(\ce{Al},\ce{Cu},\ce{Ag},\ce{Au})(ABA型六角密排等价六方)\\
\textbf{立方(cubic)}:$a=b=c$,$\alpha=\beta=\gamma=\pi/2$(bcc,fcc);\textbf{四方(tetragonal)}:$a=b\neq c$,$\alpha=\beta=\gamma=\pi/2$(bc);\textbf{正交(orthorhombic)}:$a\neq b\neq c$,$\alpha=\beta=\gamma=\pi/2$(bc,fc,base-c);\textbf{六方(hexagonal)}:$a=b\neq c$,$\alpha=\beta=\pi/2$,$\gamma=2\pi/3$;\textbf{三方(rhombohedral)}:$a=b=c$,$\alpha=\beta=\gamma\neq\pi/2$;\textbf{单斜(monoclinic)}:$a\neq b\neq c$,$\alpha=\gamma=\pi/2$,$\beta\neq\pi/2$;\textbf{三斜}:$a\neq b\neq c$,$\alpha\neq\beta\neq\gamma$\\
\rule{\columnwidth}{.2pt}\\
\textbf{Chap13倒格子,布区,晶体中的波}\\
\textbf{倒格子(reciprocal lattice)}:所有满足对任一实空间格点有$\exp(i\bm{G}\cdot\bm{R})=1$的点$\bm{G}$组成的集合,$\{\bm{G}=h\bm{b}_1+k\bm{b}_2+l\bm{b}_3\in\mathbb{R}^3\vert\exp(i\bm{k}\cdot\bm{R})=1,\forall\bm{R}\in\text{实格子}\}$;\textbf{倒格子的基矢}:$\bm{b}_1=2\pi\bm{a}_2\times\bm{a}_3/[\bm{a}_1\cdot(\bm{a}_2\times\bm{a}_3)]$,$\bm{b}_2=2\pi\bm{a}_3\times\bm{a}_1/[\bm{a}_1\cdot(\bm{a}_2\times\bm{a}_3)]$,$\bm{b}_3=2\pi\bm{a}_1\times\bm{a}_2/[\bm{a}_1\cdot(\bm{a}_2\times\bm{a}_3)]$,满足$\bm{a}_i\cdot\bm{b}_j=2\pi\delta_{ij}$;对$2$维晶格,$\bm{b}_1=2\pi\bm{a}_2\times\hat{z}/[\hat{z}\cdot(\bm{a}_1\times\bm{a}_2)]$,$\bm{b}_2=2\pi\hat{z}\times\bm{a}_1/[\hat{z}\cdot(\bm{a}_1\times\bm{a}_2)]$,其中$\hat{z}$-$\perp$晶格的单位矢量;构建倒基矢应用的实基矢,若用传统基矢,会有多余的格点\\
倒格子是实格子的傅里叶变换,证:对$1$维原子链,密度函数$\rho(r)=\sum_n\delta(r-na)$,傅变得$\mathcal{F}[\rho(r)]=\int_{-\infty}^{+\infty}\mathrm{d}r\,e^{ikr}\rho(r)=\sum_ne^{ikna}=\frac{2\pi}{a}\sum_m\delta(k-\frac{2\pi}{a}m)$,对$D$维原子链,$\mathcal{F}[\rho(\bm{r})]=\int\mathrm{d}r^3\,e^{i\bm{k}\cdot\bm{r}}\rho(\bm{r})=\sum_{\bm{R}}\int\mathrm{d}^Dr\,e^{i\bm{k}\cdot\bm{r}}\delta(\bm{r}-\bm{R})=\sum_{\bm{R}}e^{i\bm{k}\cdot\bm{R}}=\frac{(2\pi)^D}{v}\sum_{\bm{G}}\delta^D(\bm{k}-\bm{G})$,其中$v$-晶胞体积,更一般地,$\rho(\bm{r})=\rho(\bm{r}+\bm{R})$,$\mathcal{F}[\rho(\bm{r})]=\int\mathrm{d}^Dr\,e^{i\bm{k}\cdot\bm{r}}\rho(\bm{r})=\sum_{\bm{R}}\int_{\text{晶胞}}\mathrm{d}^Dr\,e^{i\bm{k}\cdot(\bm{r}+\bm{R})}\rho(\bm{r}+\bm{R})=\sum_{\bm{R}}e^{i\bm{k}\cdot\bm{R}}\int_{\text{晶胞}}\mathrm{d}^Dr\,e^{i\bm{k}\cdot\bm{r}}\rho(\bm{r})=\frac{(2\pi)^D}{v}\sum_{\bm{G}}\delta^D(\bm{k}-\bm{G})S(\bm{k})$亦得,其中\textbf{结构因子(structure factor)}$S(\bm{k})=\int_{\text{晶胞}}\mathrm{d}^Dr\,\exp(i\bm{k}\cdot\bm{r})\rho(\bm{r})$\\
\textbf{晶面(lattice plane)}:一系列等间距平行面的集合,面上包含所有原子(允许有些面上不含原子);因$\bm{G}\cdot\bm{r}=2m\pi$,每个倒格点对应一组晶面;\textbf{晶面间距}$=2\pi/G$;$\bm{G}$垂直所对应的晶面;\textbf{密勒(Miller)指数}:$(hkl)$;当密勒指数互质(最小倒格矢),对应的晶面上均有原子;对正交晶格,$\frac{2\pi}{G}=2\pi/\sqrt{h^2(2\pi/a_1)^2+k^2(2\pi/a_1)^2+l^2(2\pi/a_3)^2}$;\textbf{找$(hkl)$晶面的方法}:(不要求实基矢正交)实空间中$(1/h,0,0)$,$(0,1/k,0)$,$(0,0,1/l)$三点连成一平面,过实空间原点做另一平行面,即得\\
\rule{\columnwidth}{.2pt}\\
\textbf{Chap14晶体中的波散射}\\
\textbf{费米黄金定则}:在晶格势$V$散射(微扰)下单位时间内由分立本征态$\lvert\bm{k}\rangle$跃迁至$\lvert\bm{k}'\rangle$的概率$\Gamma(\bm{k},\bm{k}')=\frac{2\pi}{\hbar}\abs{\langle\bm{k}'\rvert V\lvert\bm{k}\rangle}^2\delta(E_{\bm{k}'}-E_{\bm{k}})$,对连续能级,单位时间跃迁概率$\int_{-\infty}^{+\infty}\Gamma(\bm{k},\bm{k}')g(E_{\bm{k}})\,\mathrm{d}E_{\bm{k}}=\frac{2\pi}{\hbar}\abs{\langle\bm{k}'\rvert V\lvert\bm{k}\rangle}^2g(E_{\bm{k}'})$,其中$E_{\bm{k}}$-$\lvert\bm{k}\rangle$的本征能,\textbf{散射密度矩阵}$M=\langle\bm{k}'\rvert V\lvert\bm{k}\rangle=\int\mathrm{d}^3r\,\frac{\exp(-i\bm{k}'\cdot\bm{r})}{\sqrt{L^3}}V(\bm{r})\frac{\exp(i\bm{k}\cdot\bm{r})}{\sqrt{L^3}}=\frac{1}{L^3}\int V(\bm{r})e^{i(\bm{k}-\bm{k}')\cdot\bm{r}}\,\mathrm{d}^3r$[散射势具周期性,$V(\bm{r}+\bm{R})=V(\bm{r})$]$=\frac{1}{L^3}\sum_{\bm{R}}\int_{\text{晶胞}}V(\bm{r}+\bm{R})\exp[i\Delta\bm{k}\cdot(\bm{r}+\bm{R})]\,\mathrm{d}^3r=\frac{1}{L^3}\sum_{\bm{R}}e^{i\Delta\bm{k}\cdot\bm{R}}\int_{\text{晶胞}}V(\bm{x})e^{i\Delta\bm{k}\cdot\bm{r}}\,\mathrm{d}^3r$;若$\Delta\bm{k}\neq\bm{G}$,$\sum_{\bm{R}}=0$,故$\Delta\bm{k}=\bm{G}$,入射与出射波矢必差一晶格动量,该动量差传递给晶体且大部分对称抵消;结构因子$\int_{\text{晶胞}}V(\bm{r})\exp(i\bm{G}\cdot\bm{r})\,\mathrm{d}^3r$影响衍射强度\\
\textbf{劳厄(Laue)公式}:$\bm{k}'-\bm{k}=\bm{G}$(动量守恒),$k=k'$(能量守恒);对称应有$\bm{k}''-\bm{k}=-\bm{G}$,但即使$\bm{k}'/\bm{k}''$两方向衍射强度不等,晶体获得动量仍很小\\
\textbf{布拉格(Bragg)公式}:$2d\sin\theta=n\lambda$,其中$d$-晶面间距,$\theta$-入/出射波矢与晶面夹角,$n$-衍射级数;两衍射公式等价,证:$\Delta k=2k\sin\theta=G=\frac{2\pi}{d}n$(劳厄)$\Leftrightarrow 2d\sin\theta=n\frac{2\pi}{k}=n\lambda$(布拉格)\\
衍射级数越高,强度越小,即倾向于小的动量变化,证:$M\propto\int_{\text{晶胞}}V(r)\exp(i\frac{2\pi}{d}nr)\,\mathrm{d}r$,$n$越大,积分值越小\\
中子衍射:中子不受库仑力作用,仅受强力作用,势函数局域于核周围,第$j$个原子的势函数$V_j(\bm{r})=f_j\delta(\bm{r}-\bm{r}_j)$,总散射势$V(\bm{r})=\sum_jV_j(\bm{r})$,故$S=\sum_{j\in\text{晶胞}}\exp(i\bm{G}\cdot\bm{r}_j)V(\bm{r})=\sum_{j\in\text{晶胞}}f_j\exp(i\bm{G}\cdot\bm{r}_j)\delta(\bm{r}-\bm{r}_j)$;X射线散射:主要与电子云相互作用,$V_j(\bm{r})=Z_jg_j(\bm{r}-\bm{r}_j)$,其中$Z_j$-原子$j$的核电荷数,$V(\bm{r})=\sum_jV_j(\bm{r}-\bm{r}_j)=\sum_{\bm{R}}\sum_{j\in\text{晶胞}}V_j(\bm{r}-\bm{R}-\bm{x}_j)$,其中$\bm{r}_j=\bm{R}+\bm{x}_j$,故$S=\int_{\text{晶胞}}\mathrm{d}^3r\,\exp(i\bm{G}\cdot\bm{r})\sum_jV_j(\bm{r}-\bm{r}_j)=\sum_{\bm{R}}\sum_{j\in\text{晶胞}}\int_{\text{晶胞}}\mathrm{d}r^3\,\exp(i\bm{G}\cdot\bm{r})V_j(\bm{r}-\bm{R}-\bm{x}_j)=\sum_{j\in\text{晶胞}}\exp(i\bm{G}\cdot\bm{x}_j)\sum_{\bm{R}}\int_{\text{晶胞}}\mathrm{d}^3r\,\exp(i\bm{G}\cdot\bm{r})V_j(\bm{r})=\sum_{j\in\text{晶胞}}\exp(i\bm{G}\cdot\bm{x}_j)\int\mathrm{d}^3r\,\exp(i\bm{G}\cdot\bm{r})V_j(\bm{r})=\sum_{j\in\text{晶胞}}\exp(i\bm{G}\cdot\bm{x}_j)f_j(\bm{G})$,其中\textbf{形状因子(form factor)}$f_j(\bm{G})=\int\mathrm{d}^3r\,\exp(i\bm{G}\cdot\bm{r})V_j(\bm{r})$,$\exp(i\bm{G}\cdot\bm{x}_j)$刻画光被同一原胞中不同原子散射的相位差;因$M$取平方,衍射结果损失失$S$中部分相位信息;X射线衍射对电荷敏感,对自旋不太敏感,便宜,中子衍射对原子核,自旋和低能激发敏感,贵;此外还有非弹性散射,中子动能小,对能量损失敏感,适用于测色散关系\\
对\ce{ScCl}(sc),$S_{(hkl)}=f_{\ce{Cs}}+f_{\ce{Cl}}\exp[i2\pi(h,k,l)\cdot(\frac{1}{2},\frac{1}{2},\frac{1}{2})]=f_{\ce{Cs}}+f_{\ce{Cl}}(-1)^{h+k+l}$;sc单质从不消光;bcc单质,$S_{(hkl)}=f\{1+\exp[i2\pi(h,k,l)\cdot(\frac{1}{2},\frac{1}{2},\frac{1}{2})]\}=f[1+(-1)^{h+k+l}]$,当$h+k+l$奇[如$(100)$],$S_{(hkl)}=0$,系统消光(Systematic Absence),满足劳厄公式却无散射,因用传统基矢构造倒格矢[$(100)$晶面未包含所有原子];对fcc单质,$S_{(hkl)}=f\{1+\exp[i2\pi(h,k,l)\cdot(\frac{1}{2},\frac{1}{2},0)]+\exp[i2\pi(h,k,l)\cdot(0,\frac{1}{2},\frac{1}{2})]+\exp[i2\pi(h,k,l)\cdot(\frac{1}{2},0,\frac{1}{2})]\}=f[1+(-1)^{h+k}+(-1)^{k+l}+(-1)^{l+h}]$,当$h,k,l$奇偶混杂时消光,当三个数奇偶性相同时不消光;对闪锌矿(\ce{ZnS},zinc blende,fcc类金刚石结构),$S_{(hkl)}=f_{\ce{Zn}}[1+(-1)^{h+k}+(-1)^{l+h}]+f_{\ce{S}}[\cdots]\exp[i\frac{\pi}{2}(h+k+l)]$,类似金刚石,$S_{(hkl)}=f_{\ce{C}}[1+\exp[i\frac{\pi}{2}(h+k+l)]][1+(-1)^{h+k}+(-1)^{k+l}+(-1)^{l+h}]$,两个中括号均可致消光,但前者所致消光并非因选错基矢[如(200)]\\
实验方法:对单晶:劳厄法(变波长)和旋转晶体法(变入射角);粉末衍射法:相当于同时扫描所有入射角,高效,制样难度小且探测器只需$1$维ccd阵列;分析方法:$2$维衍射图谱对称性确定晶体类型,峰位置确定晶胞参数,峰强确定晶胞内原子位置\\
\rule{\columnwidth}{.2pt}\\
\textbf{Part5固体中的电子Chap15周期势中的电子}\\
\textbf{近自由电子近似}:自由电子波函数$\lvert\bm{k}\rangle=\exp[i(\bm{k}\cdot\bm{x}-\omega t)]$,哈密顿$H_0=\frac{\bm{p}^2}{2m}$,本征能量$E_0(\bm{k})=\frac{\hbar k^2}{2m}$,受晶格周期性势场$V(\bm{r})=V(\bm{r}+\bm{R})$微扰,总哈密顿$H=H_0+V(\bm{r})$,薛方$H\lvert\psi\rangle=E\lvert\psi\rangle$,由微扰论,$E=E_0+E_1+E_2+\cdots$,能量$1$阶修正$E_1=\langle\bm{k}\rvert V\lvert\bm{k}\rangle=\frac{1}{L^3}\int\mathrm{d}^3r\,V(\bm{r})$为一常数,无意义,$2$阶修正$E_2=\sum_{\bm{k}'\neq\bm{k}}\frac{\abs{\langle\bm{k}'\rvert V\lvert\bm{k}\rangle}^2}{E_0(\bm{k})-E_0(\bm{k}')}$,其中$\abs{\bm{k}'-\bm{k}}=G=$($1$维情况)$\frac{2\pi}{a}n$,在非布区边界,$E_0(\bm{k})\neq E_0(\bm{k}')$,无问题,在布区边界附近,$E_0(\bm{k})\approx E_0(\bm{k})$,需用简并微扰论,$\langle\bm{k}\rvert H\lvert\bm{k}\rangle=E_0(\bm{k})(+\langle\bm{k}\rvert V\lvert\bm{k}\rangle)$,$\langle\bm{k}'\rvert H\lvert\bm{k}'\rangle=E_0(\bm{k}')(+\langle\bm{k}'\rvert V\lvert\bm{k}'\rangle)$,$\langle\bm{k}\rvert H\lvert \bm{k}'\rangle=\frac{1}{L^3}\int\mathrm{d}^3r\,\exp[i(\bm{k}'-\bm{k})\cdot\bm{r}]=\frac{1}{L^3}\int\mathrm{d}^3r\,\exp(i\bm{G}\cdot\bm{r})\equiv V_{\bm{G}}$,$\langle\bm{k}'\rvert H\lvert\bm{k}\rangle=V_{-\bm{G}}=V_{\bm{G}}^*$,设电子波函数$\lvert\psi\rangle=\alpha\lvert\bm{k}\rangle+\beta\lvert\bm{k}'\rangle$,薛方线代形式$\left[\begin{smallmatrix}
    E_0(\bm{k})&V_{\bm{G}}\\
    V_{\bm{G}}^*&E_0(\bm{k}+\bm{G})
\end{smallmatrix}\right]\left[\begin{smallmatrix}
    \alpha\\
    \beta
\end{smallmatrix}\right]=E\left[\begin{smallmatrix}
    \alpha\\
    \beta
\end{smallmatrix}\right]$,特征方程$[E_0(\bm{k})-E][E_0(\bm{k}+\bm{G})-E]-\abs{V_{\bm{G}}}^2=0$;布区边界处$\bm{k}'=-\bm{k}=$($1$维)$\frac{\pi}{a}$,$E_0(\bm{k})=E_0(\bm{k}')$,本征能$E_{\pm}=E_0(\bm{k})\pm\abs{V_{\bm{G}}}$,本征态$\lvert\psi_-\rangle=\frac{1}{\sqrt{2}}(\lvert\bm{k}\rangle+\lvert\bm{k}'\rangle)$($1$维)$=\sqrt{2}\cos\frac{\pi}{a}x[\exp(i\omega t)]$,$\lvert\psi_+\rangle=\frac{1}{\sqrt{2}}(\lvert\bm{k}\rangle-\lvert\bm{k}'\rangle)=$($1$维)$\sqrt{2}i\sin\frac{\pi}{a}x[\exp(i\omega t)]$为驻波;($1$维)偏离布区边界处,$k=\frac{\pi}{a}+\delta$,$k'=-\frac{\pi}{a}+\delta$,特征方程$\{\frac{\hbar^2}{2m}[(\frac{\pi}{a})^2+\delta^2]-E+2\frac{\hbar^2}{2m}\frac{\pi}{a}\delta\}\{\frac{\hbar^2}{2m}[(\frac{\pi}{a})^2+\delta^2]-E-2\frac{\hbar^2}{2m}\frac{\pi}{a}\delta\}-\abs{V_G}^2=0\Rightarrow\{\frac{\hbar^2}{2m}[(\frac{\pi}{a})^2+\delta^2]-E\}^2=(2\frac{\hbar^2}{2m}\frac{\pi}{a}\delta)^2+\abs{V_G}^2\Rightarrow E_{\pm}=\frac{\hbar^2}{2m}[(\frac{\pi}{a})^2+\delta^2]\pm\sqrt{(2\frac{\hbar^2}{2m}\frac{\pi}{a}\delta)^2+\abs{V_G}^2}$,当$\delta\rightarrow 0$,关于$\delta^2$展开略去高阶小量得$E_{\pm}=\frac{\hbar^2}{2m}(\frac{n\pi}{a})^2\pm\abs{V_G}+\frac{\hbar^2\delta^2}{2m}[1\pm\frac{\hbar^2}{m}(\frac{\pi}{a})^2\frac{1}{\abs{V_G}}]$\\
\textbf{布洛赫定理}:周期性势场中波函数可写为$\psi(r)=e^{i\bm{k}\cdot\bm{r}}u_{\bm{k}}(\bm{r})$,其中$u_{\bm{k}}(\bm{r})=u_{\bm{k}}(\bm{r}+\bm{G})$,证:因势场具周期性,哈密顿与平移算符对易,$[T_a,H]\psi=T_aH(r)\psi(r)-H(r)T_a\psi(r)=H(r+a)\psi(r+a)-H(r)\psi(r+a)=H(r)\psi(r+a)-H(r)\psi(r+a)=0\Rightarrow[T_a,H]=0$,故$H$与$T_a$有共同本征函数,$H\psi(r)=E\psi(r)$,$T_a\psi(r)=\psi(r+a)=\exp[i\phi(a)]\psi(r)$,$T_{a+b}\psi(r)=\exp[i\phi(a+b)]\psi(r)$,$T_aT_b\psi(r)=\exp[i\phi(a)]\exp[i\phi(b)]\psi(r)$,$T_a+T_b=T_{a+b}\Rightarrow\phi(a)+\phi(b)=\phi(a+b)\Rightarrow\phi(x)=kx$,仅当布洛赫定理成立才满足这一条件\\
\rule{\columnwidth}{.2pt}\\
\textbf{Chap16绝缘体,半导体或金属}:1维情况下,每个能级($\bm{k}$)可容$2$个电子,若每个原胞贡献$1$电子,则能带半满,外场下填充区域偏移,总动量$\neq 0$,产生电流,为导体,若每胞贡$2$,则能带全满,为\textbf{能带绝缘体};对2维情况下正方形晶格,若每胞贡$1$,在周期性势场微扰下,本为圆形的填充区域向低能的布区四边中点突出,若每胞贡$2$且带隙大,则填满方形价带,若带隙小,则在布区四边中点向导带溢出\\
\textbf{莫特绝缘体}:能带半满,但电子间交换相互作用强,电子分散排布,不愿运动到其它原子上;\textbf{安德森绝缘体}:电子运动时被散射,相干增强\\
\textbf{材料光学性质的影响因素}:1.带隙类型:\textbf{直接带隙}:价带最低点与导带最高点同$\bm{k}$,否则为\textbf{间接带隙},光子动量$\ll$布区宽度,直接带隙易保证动量守恒,易吸/发光,间接带隙吸/发光需伴随声子吸收/发射,难;2.带隙大小应$<$可见光子能量($1.5\sim 3$eV)才可高效吸光\\
能带理论中,几乎所有能级上电子自旋均一上一下成对,无法解释很多材料的磁性\\
\rule{\columnwidth}{.2pt}\\
\textbf{Chap17半导体物理}\\
\textbf{经典的半导体能带}:价带钝导带尖,因价带电子靠近原子实,束缚小,较导带电子不易跃迁,电子有效质量大于导带电子\\
\textbf{价带中电子和空穴各参数正负}:$\left\lvert\begin{smallmatrix}
    &m&p&v&E&q\\
    e&-&\pm&\pm&-&-\\
    h&+&\mp&\pm&+&+
\end{smallmatrix}\right\rvert$;\textbf{有效质量}:在布区边界附近,$E(k)=E(k_0)+\frac{1}{2}\frac{\partial^2E}{\partial k^2}k^2+\cdots\approx E_0+\frac{\hbar^2k^2}{2m^*}\Rightarrow m^*=\frac{\hbar^2}{\partial^2E/\partial k^2}$,该式不可随意套用,如对狄拉克半金属,其能带似$\times$,能量在交叉点不可幂指数展开,实际$m^*=0$,电子高速传导;\textbf{动量}:宇宙的时间反演对称要求$\sum\bm{k}=0$,当激发一$k>0$的电子,随之产生$k<0$的空穴;\textbf{能量}:空穴一般从上方开始填充,故对空穴上方能量低;\textbf{速度}:量子态对应的速度与填充的粒子种类无关,$v=\frac{\partial E}{\hbar\partial k}$\\
\textbf{载流子运动方程}(半经典):$\hbar\frac{\mathrm{d}\bm{k}}{\mathrm{d}t}=\bm{F}=q(\bm{E}+\bm{v}\times\bm{B})\approx$(价带顶/导带底处)$m^*\frac{\mathrm{d}\bm{v}}{\mathrm{d}t}$,$\bm{v}=\frac{\nabla_{\bm{k}}\epsilon}{\hbar}$;\textbf{贝瑞曲率$\bm{\Omega}$的修正}:$\hbar\frac{\mathrm{d}\bm{k}}{\mathrm{d}t}=q(\bm{E}+\bm{v}\times\bm{B})$,$\bm{v}=\frac{\nabla_{\bm{k}}\epsilon}{\hbar}+\bm{E}\times\bm{\Omega}$,这说明即使仅有恒定电场,载流子速度也不一定沿电场方向;\textbf{与德鲁德模型结合}:$\hbar\frac{\mathrm{d}\bm{k}}{\mathrm{d}t}=\bm{F}-\frac{\hbar\bm{k}}{\tau}$,$\bm{v}=\frac{\nabla_{\bm{k}}\epsilon}{\hbar}$,这说明外力直接作用于波矢,再转化到速度上;\textbf{布洛赫振荡}:电子被电场加速至一定程度后,$k$重新变为$<0$,在普通晶体中因碰撞频繁少见,在光子晶体(激光场形成的周期性结构)中可见;用载流子电性解释霍尔效应中电荷积累方向不严谨,因到底仅有电子运动,严谨地,应先看磁场对$\bm{k}$影响,再看$\bm{k}$与$\bm{v}$关系;不同材料,贡献电流的电子所处能带不同,有效质量不同,故霍尔效应中不同电荷累积方向;解释塞贝克效应亦如是\\
\textbf{本征半导体}:无掺杂的半导体;\textbf{半导体掺杂}:半导体主体一般用IV族元素,掺V/VI族元素为施主掺杂,n型半导体,电子导电,掺II/III族元素为受主掺杂,p型半导体,空穴导电;常温下多数施主原子可贡献自由电子,以\ce{P}为例,近似类氢原子,里德堡常数$R_y^*=-\frac{m^*e^4}{8\epsilon_0^2\epsilon_r^2\hbar^2}\approx(m^*\sim 0.1m_e,\epsilon_r\sim 11)\frac{13.6\text{eV}}{1000}\approx 13\text{meV}<k_BT\approx 25$meV,玻尔半径$r_0^*=\frac{4\pi\epsilon_0\epsilon_r\hbar^2}{m^*e^2}\approx 100\times 0.053\text{nm}=5$nm,不同掺杂原子的能带相对半导体主体的位置不同,降温,则掺杂原子激发产生的载流子相对半导体主体的减少更多,更趋向本征半导体\\
\textbf{本征半导体中载流子密度}:\textbf{导带上电子数密度}:$n=\int_{\epsilon_c}^{\infty}\mathrm{d}\epsilon\,g_c(\epsilon)n_F(\beta(\epsilon-\mu))$,其中$\epsilon_c$-导带底,单位体积态密度$g_c(\epsilon\geq\epsilon_c)=\frac{(2m_e^*)^{3/2}}{2\pi^2\hbar^3}\sqrt{\epsilon-\epsilon_c}$,当$k_BT\ll$带隙$\epsilon_g=\epsilon_c-\epsilon_v$,费米填充因子$n_F(\beta(\epsilon-\mu))\approx e^{-\beta(\epsilon-\mu)}$,$n\approx\frac{(2m_e^*)^{3/2}}{2\pi^2\hbar^3}\int_{\epsilon_c}^{\infty}\mathrm{d}\epsilon\,\sqrt{\epsilon-\epsilon_c}e^{-\beta(\epsilon-\mu)}=\frac{(2m_e^*)^{3/2}}{2\pi^2\hbar^3}e^{\beta(\mu-\epsilon_c)}\int_{\epsilon_c}^{\infty}\mathrm{d}\epsilon\,\sqrt{\epsilon-\epsilon_c}e^{-\beta(\epsilon-\epsilon_c)}$,设$x=\beta(\epsilon-\epsilon_c)$,$n=\frac{(2m_e^*)^{3/2}}{2\pi^2\hbar^3}e^{\beta(\mu-\epsilon_c)}\beta^{-3/2}\int_0^{\infty}\mathrm{d}x\,x^{1/2}e^{-x}=\frac{(2m_e^*)^{3/2}}{2\pi^2\hbar^3}e^{\beta(\mu-\epsilon_c)}\beta^{-3/2}\frac{\sqrt{\pi}}{2}=\frac{1}{4}(\frac{2m_e^*k_BT}{\pi\hbar^2})^{3/2}e^{-\beta(\epsilon_c-\mu)}$;\textbf{价带中空穴数密度}:$p=\int_{-\infty}^{\epsilon_v}\mathrm{d}\epsilon\,g_v(\epsilon)[1-n_F(\beta(\epsilon-\mu))]$,其中$\epsilon_v$-价带顶,单位体积态密度$g(\epsilon\leq\epsilon_v)=\frac{(2m_h^*)^{3/2}}{2\pi^2\hbar^3}\sqrt{\epsilon_v-\epsilon}$,当$k_BT\ll\epsilon_g$,$1-n_F(\beta(\epsilon-\mu))=e^{\beta(\epsilon-\mu)}$,$p=\frac{(2m_h^*)^{3/2}}{2\pi^2\hbar^3}e^{-\beta(\mu-\epsilon_v)}\int_{-\infty}^{\epsilon_v}\mathrm{d}\epsilon\,\sqrt{\epsilon_v-\epsilon}e^{-\beta(\epsilon_v-\epsilon)}=\frac{1}{4}(\frac{2m_h^*k_BT}{\pi\hbar^2})^{3/2}e^{-\beta(\mu-\epsilon_v)}$;对本征半导体,$n=p\Rightarrow\mu=\frac{\epsilon_c+\epsilon_v}{2}+\frac{3}{4}k_BT\ln\frac{m_h^*}{m_e^*}$,低温下,$\mu\approx\frac{\epsilon_c+\epsilon_v}{2}$;\textbf{两种载流子数密度之积}:$np=\frac{1}{16}(\frac{2k_BT}{\pi\hbar^2})^3(m_e^*m_h^*)^{3/2}e^{-\beta(\epsilon_c-\epsilon_v)}=\frac{1}{2}(\frac{k_BT}{\pi\hbar^2})^3(m_e^*m_h^*)^{3/2}e^{-\beta\epsilon_g}$;主体材料决定$\epsilon_g$,掺杂改变$\mu$而改变$n$,$p$,但不改变$np$,施主掺杂$\mu\uparrow$,$n\uparrow$,$p\downarrow$,受主掺杂$\mu\downarrow$,$n\downarrow$,$p\uparrow$,改变温度会改变晶格常数而改变$\epsilon_c$,$\epsilon_v$,但一般可忽略\\
\rule{\columnwidth}{.2pt}\\
\textbf{Chap18半导体器件}\\
\textbf{二极管(PN结)}:P型和N型半导体接触,电子由N极扩散至P极,产生内建电场($\sim 10^8$V/m,但不会产生电流不可用万用表量出,因电流来自费米分布的不均匀),最终抑制扩散而达平衡;在偏压$V$(P高为正)下,电流$I\propto\exp(eV/k_BT)-1$,详见半导体物理\\
\rule{\columnwidth}{.2pt}\\
\textbf{Part5磁和平均场理论Chap19原子的磁性:顺磁和逆磁}\\
磁化强度$\bm{M}=\chi\bm{H}$,其中$\chi$-磁极化率,$\bm{H}$-磁场强度;\textbf{磁性分类}:\textbf{铁(fero)磁}:$\chi\sim 10^3$,如\ce{Fe},\ce{Co},\ce{Ni},\textbf{顺(para)磁}:$\chi\sim 10^{-4}$,如\ce{Al},\textbf{抗(dia)磁}:$\chi\sim-10^{-4}$,如\ce{Cu},石墨,\textbf{常规超导}:迈斯纳效应,完全抗磁,$\chi=-1$\\
内层电子一般轨道和自旋角动量抵消,决定原子磁性的一般为外层电子,其遵循\textbf{洪特定则(Hund's rule)}:1.自旋趋同:虽磁矩同向造成磁场能量略升高,但交换相互作用主导,因为费米子,同一能级两电子总波函数$\varphi(\bm{r}_1,\bm{r}_2,\sigma_1,\sigma_2)=\psi_{\text{orbit}}(\bm{r}_1,\bm{r}_2)\chi_{\text{spin}}(\sigma_1,\sigma_2)$遵循交换反对称性(两电子交换,波函数取$-$),当自旋同向,自旋分量$\chi_{\text{spin}}=\lvert\uparrow\uparrow\rangle$或$\lvert\downarrow\downarrow\rangle$满足交换对称而轨道分量$\psi_{\text{orbit}}=(\lvert AB\rangle-\lvert BA\rangle)/\sqrt{2}$(其中$\lvert AB\rangle=A(\bm{r}_1)B(\bm{r}_2)$,$\lvert BA\rangle=A(\bm{r}_2)B(\bm{r}_1)$)交换反对称,$\psi_{\text{orbit}}(\bm{r}_1,\bm{r}_2)=-\psi_{\text{orbit}}(\bm{r}_2,\bm{r}_1)\Rightarrow\lim_{\bm{r}_1\rightarrow\bm{r}_2}\psi_{\text{orbit}}(\bm{r}_1,\bm{r}_2)=0$,即电子趋于远离,从而降低电势能,库仑相互作用能$E_{\text{singlet}}=(\langle AB\rvert-\langle BA\rvert)V(\lvert AB\rangle-\lvert AB\rangle)/2$,当自旋反向,$\chi_{\text{spin}}=\pm(\lvert\uparrow\downarrow\rangle-\lvert\downarrow\uparrow\rangle)/\sqrt{2}$,$\psi_{\text{orbit}}=(\lvert AB\rangle+\lvert BA\rangle)/\sqrt{2}$,库仑相互作用能$E_{\text{triplet}}=(\langle AB\rvert+\langle BA\rvert)V(\lvert AB\rangle+\lvert BA\rangle)/2$,交换相互作用能$E_{\text{exchange}}=E_{\text{singlet}}-E_{\text{triplet}}=-2\re\langle AB\rvert V\lvert BA\rangle$;2.轨道角动量最大化:各电子在轨道上同向转可保持相对距离较远,电势能较小,否则每转$1$圈就有$1$次靠近的机会而升高能量;3.当$<$半满,电子轨道角动量$\bm{L}$反平行自旋角动量$\bm{S}$能量低,总角动量量子数$J=\abs{L-S}$,其中$L=\sum_{\text{价电子}j}l_j$,$S=\sum_{\text{价电子}}s_j$($L$,$S$非$\bm{L}$,$\bm{S}$的模),当$>$半满,$\bm{L}$平行$\bm{S}$能量低,$J=L+S$:核产生电场$\bm{E}=\frac{Ze}{4\pi\epsilon_0r^3}\bm{r}$,核绕电子旋转产生磁场$\bm{B}=\frac{\gamma\bm{E}\times\bm{v}}{c^2}=\gamma\frac{Ze}{4\pi\epsilon_0m_ec^2r^3}\bm{r}\times\bm{p}=\gamma\frac{Ze}{4\pi\epsilon_0m_ec^2r^3}\bm{L}$,其中$\bm{v}$-电子相对核速度,相对论效应带来洛伦兹(Lorentz)因子$\gamma=[1-(\frac{v}{c})^2]^{-1/2}$,电子自旋磁矩$\bm{m}=-g\frac{\mu_B}{\hbar}\bm{S}\approx-\frac{e}{m_e}\bm{B}$,考虑托马斯进动(Thomas precession),使自旋轨道耦合$H_{\text{soc}}=-\frac{1}{2}\bm{m}\cdot\bm{B}=\frac{Ze^2}{8\pi\epsilon_0m_e^2c^2}\frac{\bm{L}\cdot\bm{S}}{r^3}$最小,其中$\bm{L}\cdot\bm{S}=\hbar\sum_jl_js_j$,即得\\
无外场下电子哈密顿$H=\frac{\bm{p}^2}{2m}+V(\bm{r})$,磁场$\bm{B}=\nabla\times\bm{A}$下哈密顿$H=\frac{(\bm{p}+e\bm{A})^2}{2m}+V(\bm{r})-\bm{B}\cdot\bm{m}=$(对均匀场可取$\bm{A}=\frac{1}{2}\bm{B}\times\bm{r}$,$[\bm{p},\bm{A}]=0$)$\frac{\bm{p}^2}{2m}+\frac{e^2}{4}\frac{\abs{\bm{B}\times\bm{r}}^2}{2m}+\frac{e\bm{p}\cdot(\bm{B}\times\bm{r})}{2m}-g\mu_B\bm{B}\cdot\bm{\sigma}+V(\bm{r})=\frac{\bm{p}^2}{2m}+\frac{e^2}{4}\frac{\abs{\bm{B}\times\bm{r}}^2}{2m}+\frac{e\bm{B}\cdot(\bm{r}\times\bm{p})}{2m}-g\mu_B\bm{B}\cdot\bm{\sigma}+V(\bm{r})=\frac{\bm{p}^2}{2m}+\frac{e^2}{4}\frac{\abs{\bm{B}\times\bm{r}}^2}{2m}+\mu_B\bm{B}\cdot(\bm{L}-g\bm{\sigma})+V(\bm{r})$,其中第$2$项-逆磁项,来自轨道运动,第三项-顺磁项,来自自旋轨道耦合\\
\textbf{居里(Curie)顺磁}:对自由电子,平均磁矩$m=\frac{(g\mu_B/2)\exp(\beta g\mu_BB/2)+(-g\mu_B/2)\exp(-\beta g\mu_BB/2)}{\exp(\beta g\mu_BB/2)+\exp(-\beta g\mu_BB/2)}=(g\mu_B/2)\tanh(\beta g\mu_BB/2)$,或配分函数$Z=e^{\beta g\mu_BB/2}+e^{-\beta g\mu_BB/2}$,自由能$F=-k_BT\ln Z$,$m=-\frac{\partial F}{\partial B}$得,磁化强度$M=nm$,其中$n$-电子数密度,磁化率$\chi_c=\lim_{H\rightarrow 0}\frac{\partial M}{\partial H}=\lim_{B\rightarrow 0}\frac{\mu_0\partial M}{\partial B}=n\frac{\mu_0(g\mu_B)^2}{4k_BT}$(\textbf{居里定律}),用了玻尔兹曼分布,故与$T$有关;\textbf{原子中电子朗德$g$因子}:$\tilde{g}=\frac{1}{2}(g+1)+\frac{1}{2}(g-1)\frac{S(S+1)-L(L-1)}{J(J+1)}$,从而顺磁项$=-\tilde{g}\mu_B\bm{B}\cdot\bm{J}$,$\chi_c=n\frac{\mu_0(\tilde{g}\mu_B)^2}{3}\frac{J(J+1)}{k_BT}$\\
\textbf{拉莫尔(Larmor)逆磁}:设$\bm{B}=B\hat{z}$,自由电子逆磁项$E=\frac{e^2}{8m}B^2\langle x^2+y^2\rangle=\frac{e^2}{8m}B^2\frac{2}{3}\langle r^2\rangle$,平均磁矩$\langle\bm{m}\rangle=-\frac{\partial E}{\partial\bm{B}}=-\frac{e^2}{6m}\langle r^2\rangle\bm{B}$,磁化率$\chi_l=\lim_{B\rightarrow 0}n\mu_0\frac{\partial m}{\partial B}=-n\frac{e^2\mu_0\langle r^2\rangle}{6m}$;固体中,$\chi_l=-n\frac{Ze^2\mu_0\langle r^2\rangle}{6m}$;\textbf{朗道(Landau)逆磁}:源于磁场中巡游电子旋转,$\chi_{\text{Landau}}=-\frac{1}{3}\chi_p$\\
$\frac{\chi_p}{\chi_c}=\frac{g(\epsilon_F)k_BT}{n}$,$3$维情况态密度$g(\epsilon)=\alpha\sqrt{\epsilon}$,非高温下$\int_0^{\epsilon_F}\alpha\sqrt{\epsilon}\,\mathrm{d}\epsilon\approx n\Rightarrow\frac{2}{3}\alpha\epsilon^{3/2}=n\Rightarrow g(\epsilon_F)=\frac{3n}{2\epsilon_F}$,故$\frac{\chi_p}{\chi_c}=\frac{3k_BT}{2\epsilon_F}=\frac{3k_BT}{2k_BT_F}\ll 1$,居里顺磁远显著于泡利顺磁\\
\rule{\columnwidth}{.2pt}\\
\textbf{Chap20自发磁有序:铁磁,反铁磁和铁氧化物磁性}\\
\textbf{海森堡(Heisenberg)模型}:适用于绝缘体,自旋被束缚于晶格中,总哈密顿$H=-\frac{1}{2}\sum_{ij}J_{ij}\bm{S}_i\cdot\bm{S}_j+\sum_ig\mu_B\bm{B}\cdot\bm{S}_i$,若仅考虑近邻自旋相互作用,$H=-\frac{1}{2}\sum_{\langle i,j\rangle}J_{ij}\bm{S}_i\cdot\bm{S}_j+\sum_ig\mu_B\bm{B}\cdot\bm{S}_i$;\textbf{伊辛(Ising)模型}:假设自旋仅有上下两种取向,$H=-\frac{1}{2}\sum_{\langle i,j\rangle}J_{ij}S_iS_j+\sum_ig\mu_BB_zS_i$,其中$S_i=\pm s$,对均匀介质,$J_{ij}=J\forall\langle i,j\rangle$;无外场下,若$J<0$,趋向$\bm{S}_i\parallel\bm{S}_j$,长程磁有序,\textbf{铁磁},若$J<0$,趋向$\bm{S}_i$反$\parallel\bm{S}_j$,相邻自旋反平行,反铁磁;可通过中子($s=1/2$)衍射判断反铁磁,相邻反平行使中子感受到的晶格常数是传统晶格常数的$2$倍;三角晶格中$J<0$导致阻错(frustrated);\textbf{亚铁磁}:相邻反平行,但朝两个方向的磁矩大小不同\\
\textbf{对称性破缺}:自旋因环境的不对称而呈各项异性,如当总哈密顿$H=-\frac{1}{2}\sum_{\langle i,j\rangle}J\bm{S}_i\cdot\bm{S}_j-\kappa\sum_i(S_i^z)^2$,自旋趋向平行$z$轴\\
\rule{\columnwidth}{.2pt}\\
\textbf{Chap21磁畴和磁滞}\\
\textbf{磁畴}:具同向磁矩的小区域,这些小区域由于大量磁矩的长程相互作用形成;\textbf{畴壁}:磁畴间的过渡区域\\
设畴壁厚$Na$,每两个相邻自旋夹角$\delta\theta=\pi/N$,交换相互作用能$E=-J\bm{S}_i\cdot\bm{S}_j=-JS^2\cos\delta\theta=-JS^2[1-\frac{(\delta\theta)^2}{2}+\cdots]$,各向异性能$=\kappa\sum_{j=1}^{N-1}(S\cos j\delta\theta)^2\approx N\kappa S^2/2$,相较无畴壁总能量变化$\delta E=JS^2(\pi^2/2)/N+N\kappa S^2/2$,取$N\approx\pi\sqrt{J/\kappa}$得最小能量$\delta E_{\min}=\pi S^2\sqrt{J\kappa}/2$,故$J/\kappa$越大畴壁越宽\\
\textbf{磁滞}:\textbf{剩磁}:撤去磁场$H$后仍有的磁感应强度,$B(H=0)$;\textbf{矫顽力}:消除剩磁所需的磁场强度,$H(B=0)$;\textbf{磁能积}:$BH$;一般磁性越强,工作温度越低;磁滞的原理:设各向异性轴为$z$轴,外加磁场$\bm{B}=B\hat{z}$,单位体积能量$E/V=E_0-\bm{M}\cdot\bm{B}-\kappa'M_z^2=E_0-MB\cos\theta-\kappa'M^2\cos^2\theta$,其中$\theta$-$\bm{M}$与$\bm{B}$夹角,当$B>B_{\text{crit}}=2\kappa'M$,$E/V$在$\cos\theta\in[-1,1]$区间$\downarrow$,为使能量最小,只能$\cos\theta=1$,即$\bm{M}\parallel\bm{B}$,$M-B$曲线水平,当$0<B<B_{\text{crit}}$,$E/V$在$\cos\theta\in[-1,1]$区间有两极值点$\cos\theta=-1$和$1$,其中$\cos\theta=-1$并非最小值,故$\bm{M}$反$\parallel\bm{B}$为一亚稳态,对应磁滞现象;当$\bm{B}$与各项异性轴成夹角$\phi$,$E/V=E_0-MB\cos\theta-\kappa'M^2\cos^2(\phi-\theta)$,当$B$强,第$2$项主导,$\theta\rightarrow 0$,$\bm{M}\parallel\bm{B}$,当$B$弱,第$3$项主导,$\theta\rightarrow\phi$,$\bm{M}\parallel$各向异性轴;当$\bm{B}\perp$各向异性轴,$E/V=E_0-MB\cos\theta-\kappa'M^2\sin^2\theta=E_0-MB\cos\theta+\kappa'M^2\cos^2\theta-\kappa'M^2$,$\forall B\exists$唯一极小值,当$0<B<B_{\text{crit}}=2\kappa'M$,极小值点$\cos\theta=\frac{B}{2\kappa'M}$,$M-B$曲线$\uparrow$,当$B>B_{\text{crit}}=2\kappa'M$,极小值点$\cos\theta=1$,$M-B$曲线水平;对一块普通的铁磁性材料,因其中有许多磁畴,其$M-B$图介于上述两种极端情况之间\\
冰箱贴一面吸引铁磁性物质,一面无法吸引,因其中存在海尔贝克阵列的磁畴,如$\uparrow\rightarrow\downarrow\leftarrow\uparrow\cdots$,上方磁场叠加减弱,下方增强,此种阵列还可用于自由电子激光所需的波荡器\\
小磁针可视为一磁畴,其各向异性由地磁场提供,$E=-\bm{B}_{\text{earth}}\cdot\bm{M}$\\
\textbf{磁各向异性的来源}:1.单晶(磁畴)本身的各向异性,如晶格形状,可施加应力影响;2.宏观形状,柱状物体沿母线更易被磁化\\
\rule{\columnwidth}{.2pt}\\
\textbf{Chap22平均场理论}\\
对伊辛模型中的两个自旋,哈密顿$H=-J\sigma_1\sigma_2+g\mu_BB(\sigma_1+\sigma_2)$,求解流程:1.求本征能谱$\{\epsilon_i\}$,2.写配分函数,3.算各种物理量;取本征矢$\lvert 1\rangle=\lvert\uparrow\uparrow\rangle$,$\lvert 2\rangle=\lvert\uparrow\downarrow\rangle$,$\lvert 3\rangle=\lvert\downarrow\uparrow\rangle$,$\lvert 4\rangle=\lvert\downarrow\downarrow\rangle$,无外场时,$\langle 1\rvert H\lvert 1\rangle=-J$,$\langle 1\rvert H\lvert 2\rangle=0$,$H=\left[\begin{smallmatrix}
    -J&0&0&0\\
    0&J&0&0\\
    0&0&J&0\\
    0&0&0&-J
\end{smallmatrix}\right]$,本征能量:$\epsilon=-J,J,J,-J$,对$N$个自旋的体系,$\dim(H)=2^N\times 2^N$,计算极为复杂困难\\
\textbf{平均场近似}:第$i$个自旋的哈密顿$H_i=(-J\sum_{j\text{ near }i}\sigma_j-g\mu_BB)\sigma_i=-g\mu_BB_{\text{eff}}\sigma_i$,其中等效磁场满足$g\mu_BB_{\text{eff}}=J\sum_{j\text{ near i}}\sigma_j+g\mu_BB=Jz\langle\sigma\rangle+g\mu_BB$,其中$z$-相邻自旋数,平均磁矩自洽方程$\langle\sigma\rangle=\frac{(1/2)\exp(\beta g\mu_BB_{\text{eff}}/2)-(1/2)\exp(-\beta g\mu_BB_{\text{eff}}/2)}{\exp(\beta g\mu_BB_{\text{eff}}/2)+\exp(-\beta g\mu_BB_{\text{eff}}/2)}=\frac{1}{2}\tanh(\beta g\mu_BB_{\text{eff}}/2)=\frac{1}{2}\tanh[\beta(Jz\langle\sigma\rangle+g\mu_BB)/2]$;无外场时,$\langle\sigma\rangle=\frac{1}{2}\tanh(\beta Jz\langle\sigma\rangle/2)=y(\langle\sigma\rangle)$,$y(0)'=\beta Jz/4$,当$y(0)'<1$即$T>$\textbf{居里温度}$T_c=\frac{Jz}{4k_B}$时,仅有$\langle\sigma\rangle=0$,各向同性,当$y(0)'>1$即$T<T_c$时,$\langle\sigma\rangle$有非零解,自发磁有序,自发性对称性破缺;高温下,自洽方程近似为$\langle\sigma\rangle\approx\frac{1}{4}\beta(Jz\langle\sigma\rangle+g\mu_BB)\Rightarrow\langle\sigma\rangle=\frac{\beta g\mu_BB/4}{1-\beta Jz/4}=\frac{g\mu_BB/4}{k_B(T-T_c)}$,其中临界温度$T_c=\frac{Jz}{4k_B}$,磁化率$\chi=n\mu_0\frac{\mathrm{d}(g\mu_B\langle\sigma\rangle)}{\mathrm{d}B}=n\frac{\mu_0(g\mu_B)^2/4}{k_B(T-T_c)}=\frac{\chi_c}{1-T_c/T}$(\textbf{居里-韦斯(Weiss)定律});铁磁情况,$T_c>0$,反铁磁情况,$T_c<0$;平均场理论的缺陷:无法解释1维材料无$T_c$\\
\rule{\columnwidth}{.2pt}\\
\textbf{Chap23来自相互作用的磁:哈伯德模型}:在伊辛模型基础上额外考虑自由电子的动能,哈密顿$H=H_0+H_I$,其中$H_0$-紧束缚模型的哈密顿,自旋相互作用哈密顿$H_I=u\sum_in_{i\uparrow}n_{i\downarrow}$,$n_{i\uparrow/\downarrow}$-第$i$个原子上的$\uparrow/\downarrow$自旋数,这意味着仅当两个电子在同一原子上才有相互作用,从中可推出巡游(itinerant)磁性,且为使能量最小,自旋同向的电子不处于同一原子,$un_{i\uparrow}n_{i\downarrow}=\frac{u}{4}[(n_{i\uparrow}+n_{i\downarrow})^2-(n_{i\uparrow}-n_{i\downarrow})^2]\approx\frac{u}{4}[\langle n_{i\uparrow}+n_{i\downarrow}\rangle^2-(\frac{Mv}{g\mu_B/2})^2]$,其中首项$\propto$总电子数,$v$-原胞体积;若翻转少量自旋,动能变化$\Delta E_k=\frac{V}{2}g(\epsilon_F)\cdot\frac{1}{2}\delta E\cdot\frac{1}{2}\delta E=\frac{V}{8}g(\epsilon_F)(\delta E)^2$,磁化强度变为$M=\frac{1}{2}g(\epsilon_F)\cdot\frac{1}{2}\delta E\cdot 2g\mu_B/2=\frac{1}{4}g(\epsilon_F)g\mu_B\delta E$,交换相互作用能变化$\abs{\Delta E_I}=\frac{V}{v}\frac{u}{4}(\frac{Mv}{g\mu_B/2})^2=\abs{\Delta E_k}\Rightarrow u=\frac{2}{vg(\epsilon_F)}$,当$u>\frac{2}{vg(\epsilon_F)}$,自发磁有序,铁磁,当$u>\frac{2}{vg(\epsilon_F)}$,无磁性(\textbf{斯托纳判据(Stoner criterion)})\\
\textbf{莫特绝缘}:当$u$继续增大,同一原子上自旋反向的能量很高,此时相邻自旋相同,导致电子无法运动;\textbf{莫特反铁磁}:相邻自旋相反时电子有可能跃迁,波函数分散,能量更低,反铁磁
\end{multicols}
\begin{multicols}{4}
\noindent\textbf{一些常数}\\
\textbf{玻尔兹曼常数}:$k_B=1.38\times 10^{-23}\text{J}/\text{K}$\\
\textbf{理想气体常数}:$R=8.31\text{J}\cdot\text{mol}^{-1}\cdot\text{K}$\\
\textbf{普朗克常数}:$h=6.63\times 10^{-34}\text{J}\cdot\text{s}$\\
\textbf{阿伏伽德罗常数}:$N_A=6.02\times 10^{-23}\text{mol}^{-1}$\\
$0\text{C}^{\circ}=273.15$K\\
\textbf{元电荷}:$e=1.60\times 10^{-19}$C\\
\textbf{电子质量}:$m_e=9.11\times 10^{-31}\text{kg}=0.511$M$e$V\\
\textbf{玻尔磁子}:$\mu_B=9.27\times 10^{-24}\text{J}/\text{T}$\\
\textbf{真空中磁导率}:$\mu_0=4\pi\times 10^{-7}\text{N}/\text{A}^2$\\
\textbf{真空中介电常数}:$\epsilon_0=8.85\times 10^{-12}\text{F}/\text{m}$\\
\textbf{大气压}:$1.01\times 10^5$Pa
\end{multicols}
\end{document}